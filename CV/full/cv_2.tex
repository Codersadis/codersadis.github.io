%%%%%%%%%%%%%%%%%%%%%%%%%%%%%%%%%%%%%%%%%
% Medium Length Graduate Curriculum Vitae
% LaTeX Template
% Version 1.1 (9/12/12)
%
% This template has been downloaded from:
% http://www.LaTeXTemplates.com
%
% Original author:
% Rensselaer Polytechnic Institute (http://www.rpi.edu/dept/arc/training/latex/resumes/)
%
% Important note:
% This template requires the res.cls file to be in the same directory as the
% .tex file. The res.cls file provides the resume style used for structuring the
% document.
%
%%%%%%%%%%%%%%%%%%%%%%%%%%%%%%%%%%%%%%%%%

%----------------------------------------------------------------------------------------
%	PACKAGES AND OTHER DOCUMENT CONFIGURATIONS
%----------------------------------------------------------------------------------------

\documentclass[margin, 10pt]{res} % Use the res.cls style, the font size can be changed to 11pt or 12pt here

\usepackage{helvet} % Default font is the helvetica postscript font
%\usepackage{newcent} % To change the default font to the new century schoolbook postscript font uncomment this line and comment the one above
\usepackage{hyperref}
\usepackage{color}
\usepackage{fancyhdr}
\usepackage{textcomp}
\usepackage{url}

\usepackage{marvosym}
\usepackage{pifont}

\setlength{\textwidth}{5.1in} % Text width of the document

\begin{document}

%----------------------------------------------------------------------------------------
%	NAME AND ADDRESS SECTION
%----------------------------------------------------------------------------------------

\moveleft.5\hoffset\centerline{\large\bf Chuanmin Jia} % Your name at the top

\moveleft\hoffset\vbox{\hrule width\resumewidth height 1pt}\smallskip % Horizontal line after name; adjust line thickness by changing the '1pt'

\moveleft.5\hoffset\centerline{Science Building 2728, Peking University} % Your address
\moveleft.5\hoffset\centerline{Haidian District, Beijing 100871, P.R.China}
\moveleft.5\hoffset\centerline{\Letter~\href{mailto:cmjia@pku.edu.cn}{\texttt{cmjia@pku.edu.cn}}
, \Letter~\href{mailto:jiacm@jdl.ac.cn}{\texttt{jiacm@jdl.ac.cn}}}
\moveleft.1\hoffset\centerline{}
%\moveleft.5\hoffset\centerline{(+86) 188- or (111) 111-1112}

%----------------------------------------------------------------------------------------

\begin{resume}

%----------------------------------------------------------------------------------------
%	OBJECTIVE SECTION
%----------------------------------------------------------------------------------------

\section{RESEARCH INTERESTS}

\begin{itemize} \itemsep -2pt % Reduce space between items
\item{Video Compression/Processing}
\item{Deep feature coding}
\item{Machine Learning}
\end{itemize}

%----------------------------------------------------------------------------------------
%	EDUCATION SECTION
%----------------------------------------------------------------------------------------

\section{EDUCATION}
% 1
{\bf Peking University (PKU)}, BJ, CHN, \hfill{ 2015.Sep~--~present}
\begin{itemize} \itemsep -1pt
\item{{\sl Ph.D. student,} Electronics Engineering \& Computer Science}
\item{Advisor: Prof. Siwei Ma and Prof. Wen Gao}
\end{itemize}
% 2
{\bf New York University (NYU)}, NY, USA, \hfill{ 2017.Dec~--~present}
\begin{itemize} \itemsep -1pt
\item{{\sl Visiting Ph.D student,} Electronic and Computer Engineering}
\item{Advisor: Prof. Yao Wang}
\end{itemize}
% 3
{\bf Beijing Univ. of Posts. \& Telecom. (BUPT)}, BJ, CHN, \hfill{ 2011.Sep~--~2015.July}
\begin{itemize} \itemsep -1pt
\item{{\sl B.Eng,} School of Computer Science}
\item{GPA: 86.7/100, rank: 35/301}
\item{Thesis: Research on Compressed Video Enhancement and GPU Acceleration.}
\end{itemize}


%----------------------------------------------------------------------------------------
%	RESEARCH EXPERIENCE SECTION
%----------------------------------------------------------------------------------------

\section{RESEARCH EXPERIENCE}

{\sl Visiting scholar}, NYU-Tandon \hfill {Dec. 2017 -- present} \\
Video Lab, Brooklyn, NY

\begin{itemize} \itemsep -2pt % Reduce space between items
\item Research on deep learning based coding tools for next generation video coding standard.
\item Deep learning feature coding algorithms for facial images and surveillance videos.
\end{itemize}

{\sl Research Assistant}, PKU-EECS \hfill {Sep. 2014 -- present} \\
Institute of Digital Media, Beijing

\begin{itemize} \itemsep -2pt
\item{Designed machine learning based in-loop filtering video coding tools for future video coding standards.}
\item{Implemented video restoration and quality enhancement algorithm based on non-local self similarity prior.}
\item{Proposed high efficiency light field image compression algorithm based on sub-aperture adaptation.}
\item{Optimized virtual-view synthesis algorithm using CUDA, achieved real-time view synthesis for full HD videos.}
\end{itemize}

{\sl Research Intern}, PKU-EECS \hfill {Feb. 2014 -- Aug. 2014} \\
Institute of Computational Linguistics, Beijing
\begin{itemize}  \itemsep -2pt
\item{Conducted performance comparison on different deep learning algorithms for Chinese word segmentation and word embedding.}
\end{itemize}

{\sl Research Intern}, BUPT-SCS \hfill {Aug. 2013 -- Mar. 2014} \\
Innovation Center, Beijing
\begin{itemize}
\item{Interned as a national undergraduate projects member for innovation research.}
\end{itemize}

%----------------------------------------------------------------------------------------
%	PUBLICATION SECTION
%----------------------------------------------------------------------------------------

\section{PUBLICATIONS}
{\sl Journal Papers}
\begin{itemize}
\item{S. Ma, S. Wang, X. Zhang, X. Zhang and \underline{\bf C. Jia}, ``Joint Feature and Texture Coding: Towards Smart Video Representation via Front-end Intelligence," {\bf submitted} to IEEE Transactions on Circuits and Systems for Video Technology (TCSVT), 2018 (Under Review).}
\item{\underline{\bf C. Jia}, F. Luo, X. Zhang, S. Wang, S. Wang and S. Ma, ``Fast Non-local Adaptive In-Loop Filter Optimization on GPU," {\bf submitted} to IEEE Transactions on Circuits and Systems for Video Technology (TCSVT), 2018 (Under Review).}
\item{\underline{\bf C. Jia}, S. Wang, X. Zhang, S. Wang, J. Liu and S. Ma, ``Content-Aware Convolutional Neural Network for In-loop Filtering in High Efficiency Video Coding," {\bf submitted} to IEEE Transactions on Image Processing (TIP), 2017 (Under Review).}
\item{S. Ma, X. Zhang, J. Zhang, \underline{\bf C. Jia}, S. Wang and W. Gao ``Nonlocal In-Loop Filter: The Way Toward Next-Generation Video Coding?," {\em {IEEE} MultiMedia} 23 (2), 16-26.}
\end{itemize}

{\sl Conference Papers}
\begin{itemize}
\item{Y. Yang, Z. Zhao, \underline{\bf C. Jia}, X.Zhang, S. Wang and S. Ma, ``Convolutional Neural Network based Intermediate View Synthesis for Light Field Image Compression," {\bf accepted} by {\em IEEE International Workshop on Multimedia Signal Processing ({\bf MMSP})}, Vancouver, Canada, Aug, 2018. }

\item{S. Wang, Z. Zhao, \underline{\bf C. Jia}, X. Zhang, X. Zhang, S. Wang, S. Ma and W. Gao, ``Deep Network Based Image Compression with Adaptive Pre- and Postprocessing," {\bf accepted} by {\em IEEE International Workshop on Multimedia Signal Processing ({\bf MMSP})}, Vancouver, Canada, Aug, 2018. }

\item{X. Meng, \underline{\bf C. Jia}, S. Wang, X. Zheng and S. Ma, ``Optimized Non-local In-Loop Filter for Video Coding," {\bf accepted} by {\em IEEE Picture Coding Symposium ({\bf PCS})}, Los Angeles, California, USA, June, 2018. (Poster)}

\item{Z. Zhao, S. Wang, \underline{\bf C. Jia}, X. Zhang, S. Ma and J. Yang, ``Light Field Image Compression Based on Deep Learning," {\bf accepted} by {\em IEEE International Conference on Multimedia \& Expo ({\bf ICME})}, San Diego, California, USA, July, 2018. (Oral, 15\%)}

\item{Y. Wang, X. Fan, \underline{\bf C. Jia}, D. Zhao and W. Gao, ``Neural Network Based Inter Prediction for HEVC," {\bf accepted} by {\em IEEE International Conference on Multimedia \& Expo ({\bf ICME})}, San Diego, California, USA, July, 2018. (Poster, 30\%)}

\item{\underline{\bf C. Jia}, S. Wang, X. Zhang, S. Wang and S. Ma, ``Spatial-Temporal Residue Network Based In-Loop Filter for Video Coding," {\em Proc. of IEEE Visual Communications and Image Processing ({\bf VCIP})}, St.Petersburg, Florida, USA, Dec, 2017. (Oral)}

\item{\underline{\bf C. Jia}, Y. Yang, X. Zhang, S. Wang, S. Wang and S. Ma, ``Light Field Image Compression with Sub-apertures Reordering and Adaptive Reconstruction," {\em Proc. of the Pacific-Rim Conference on Multimedia ({\bf PCM}}), Harbin, China, Sept, 2017. (Oral)} ({\bf{\color{red} Best Paper Award}})

\item{\underline{\bf C. Jia}, Y. Yang, X. Zhang, S. Wang, X. Zhang, S. Wang and S. Ma, ``Optimized Inter-view Prediction Based Light Field Image Compression with Adaptive Reconstruction," {\em Proc. of IEEE International Conference on Image Processing ({\bf ICIP}}), grand challenge for LF image coding, Beijing, China, Sept, 2017. (Oral)}

\item{\underline{\bf C. Jia}, X. Zhang, J. Zhang, S. Wang and S. Ma, ``Deep Convolutional Network based Image Quality Enhancement for Low Bit Rate Image Compression," {\em Proc. of IEEE Visual Communications and Image Processing ({\bf VCIP})}, Chengdu, China, Nov. 2016. (Oral)}

\item{J. Zhang, \underline{\bf C. Jia},  N. Zhang, S. Ma, and W. Gao, ``Structure-driven Adaptive Non-local Filter for High Efficiency Video Coding (HEVC)," {\em Proc. of IEEE Data Compression Conference ({\bf DCC})}, Snowbird, Utah, USA, Mar. 2016. (Oral) ({\bf Top Conference in Data Compression})}

\item{J. Zhang, \underline{\bf C. Jia}, S. Ma, and W. Gao, ``Non-Local Structure-Based Filter for Video Coding," {\em Proc. of IEEE International Symposium on Multimedia ({\bf ISM})}, Miami, Florida, USA, Dec. 2015. (Oral)}

\end{itemize}

{\sl Standardization Contributions}
\begin{itemize}
\item{Z. Wang, X. Meng, \underline{\bf C. Jia}, J. Cui, S. H. Wang, S. Wang, S. Ma, W. Li, Z. Miao and X. Zheng, ``Description of SDR video coding technology proposal by DJI and Peking University," Joint Video Exploration Team (JVET) of ITU-T SG, {\bf JVET-J0011}, San Diego, USA, April, 2018. }
\item{X. Meng, \underline{\bf C. Jia}, Z. Wang, S. Wang, S. Ma, X. Zheng, ``Non-local Structure-based Filter with integer operation," Joint Video Exploration Team (JVET) of ITU-T SG, {\bf JVET-J0071}, San Diego, USA, April, 2018.}

\end{itemize}


%----------------------------------------------------------------------------------------
%	COMMUNITY SERVICE SECTION
%----------------------------------------------------------------------------------------

\section{PROFESSIONAL \\ ACTIVITY}

Reviewer Service
\begin{itemize} \itemsep -2pt
\item{Journal of Visual Communication and Image Representation (JVCIR). }
\item{IEEE International Conference on Image Processing (ICIP).}
\item{IEEE International Conference on Multimedia and Expo (ICME). }
\item{IEEE International Symposium on Multimedia (ISM). }
\item{IEEE Visual Communication and Image Processing (VCIP). }
\item{IEEE Student Member}
\end{itemize}

Conference Presentations and Invited Talks
\begin{itemize} \itemsep -2pt
\item{Description of SDR video coding technology proposal by DJI and Peking University, {\em San Diego, CA, U.S, April. 2018}}
\item{Non-local Structure-based Filter with integer operation, {\em San Diego, CA, U.S, April. 2018}}
\item{Spatial-Temporal Residue Network Based In-Loop Filter for Video Coding, {\em VCIP2017, St Petersburg, FL, U.S, Dec. 2017}}
\item{Light Field Image Compression with Sub-apertures Reordering and Adaptive Reconstruction, {\em PCM2017, Harbin, China, Sep. 2017}}
\item{Optimized Inter-View Prediction Based Light Field Image Compression With Adaptive Reconstruction, {\em ICIP2017, Grand Challenge for Light Field Image coding, Beijing, China, Sep. 2017}}
\item{Deep Convolutional Network based Image Quality Enhancement for Low Bit Rate Image Compression, {\em VCIP2016, Chengdu, China, Nov. 2016}}
\end{itemize}

%----------------------------------------------------------------------------------------
%	TEACHING EXPERIENCE
%----------------------------------------------------------------------------------------

\section{TEACHING \\ EXPERIENCE}

TA: Video Coding and Understanding (EECS 04812102), EECS, PKU, \hfill Spring.2017 \\
TA (for projects): Image and Video Processing (EL-GY 6123), ECE, NYU, \hfill Spring.2018 \\


%----------------------------------------------------------------------------------------
%	COMPUTER SKILLS SECTION
%----------------------------------------------------------------------------------------

\section{COMPUTER \\ SKILLS}

{\sl Languages \& Software:}
C/C++, CUDA, MATLAB, Power Shell, Python, \LaTeX. \\
{\sl Operating Systems:}
Mac OS X, Ubuntu Linux, Windows. \\
{\sl Libraries/Frameworks:}
Caffe, MXNET, Tensorflow, HM, AVS2, JEM. \\
{\sl Github Repo:}
\Pickup~{\href{https://github.com/codersadis}{\texttt{https://github.com/codersadis}}} \\
{\sl Homepage:}
\Pickup~{\href{http://http:www.jiachuanmin.site}{\texttt{http://www.jiachuanmin.site}}} \\
{\sl Google Scholar:}
\Pickup~{\href{https://scholar.google.com/citations?user=x5Na9n0AAAAJ}{\texttt{https://scholar.google.com/citations?user=x5Na9n0AAAAJ}}} \\


%----------------------------------------------------------------------------------------
%	SELECTED \& PROJECTS
%----------------------------------------------------------------------------------------

%\section{SELECTED \\ PROJECTS}
%View Synthesis Optimization, \hfill{Apr. 2016 - Sep. 2016}
%\begin{itemize} \itemsep -2pt
%    \item{Optimized an open-source virtual-view synthesis software using CUDA. }
%    \item{Achieved 6x acceleration, with real time full HD (1080P) view synthesis over 40fps.}
%    \item{Implemented left view and right view wrapping, blending and fill occlusion parallelism algorithm in CUDA.}
%\end{itemize}
%
%NFC Tour Guide, \hfill{Aug. 2013 - Mar. 2014}
%\begin{itemize} \itemsep -2pt
%    \item{Developed an Android app using NFC for tourism}
%    \item{Mainly responsible for implementing of NFC pay, speech tour guide and database interface design.}
%\end{itemize}
%Flower Recognition, \hfill{Jul. 2013 - Sep. 2013}
%\begin{itemize} \itemsep -2pt
%    \item{Proposed flower recognition algorithm by combining histogram and contour feature with linear classifier.}
%    \item{Implemented iOS app development and recognition algorithm APIs.}
%\end{itemize}


%----------------------------------------------------------------------------------------
%	HONORS \& AWARD
%----------------------------------------------------------------------------------------

\section{HONORS \& AWARDS}
{\bf Outstanding Reviewer} of JVCIR, \hfill 2017 \\
{\bf Best Reviewer} of IEEE Visual Communication and Image Processing (VCIP), \hfill 2017 \\
{\bf Best Paper Award} of Pacific-Rim Conference on Multimedia (PCM), \hfill 2017 \\
{\bf Outstanding Reviewer} of JVCIR, \hfill 2016 \\
{\bf ${1}^{st}$} prize of Video Big Data Compression Contest of National Graduate Contest on Smart-City Technology. \hfill 2016 \\
Excellent Graduation Thesis Award, BUPT, \hfill 2015 \\
Excellent Undergraduates, BUPT, \hfill 2015 \\
Innovation Scholarship, PKU, \hfill 2015 \\
Honorable Mention Winner in Mathematical Contest in Modeling (MCM), \hfill 2014 \\

%----------------------------------------------------------------------------------------

\moveleft.5\hoffset\centerline{
\textit{Last updated:}
\today
}

\end{resume}
\end{document}
