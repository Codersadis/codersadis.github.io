
\documentclass[margin,line,sans,10pt]{res}

\newcommand{\myfont}{\fontsize{11pt}{11pt}\selectfont}

%\renewcommand{\ttdefault}{pcr}
\renewcommand*{\namefont}{\Huge\bfseries\upshape}
\renewcommand*{\sectionfont}{\myfont}

\RequirePackage{tgpagella}
\renewcommand*{\familydefault}{\rmdefault}
% yulunzhang's code
\usepackage{fancyhdr}
\usepackage{textcomp}
\pagestyle{plain}
\pagenumbering{arabic}
% yulunzhang's code
\usepackage{hyperref}
%\hypersetup{
%   colorlinks=true,       % false: boxed links; true: colored links
%   linkcolor=red,          % color of internal links
%   citecolor=green,        % color of links to bibliography
%   filecolor=magenta,      % color of file links
%   urlcolor=blue,           % color of external links
%}

%\hypersetup{pdfborderstyle={/S/U/W 1}}
%\hypersetup{pdfborderstyle={/S/D/D[3 2]/W 1}}

\usepackage{url}

\usepackage{marvosym}
\usepackage{pifont}

\hyphenpenalty=750

\oddsidemargin -.41in \evensidemargin -1in \textwidth=5.7in
\textheight 9in
\itemsep=0in
\parsep=0in

\newenvironment{list1}{
  \begin{list}{\ding{113}}{%
      \setlength{\itemsep}{0.05in}
      \setlength{\parsep}{0in} \setlength{\parskip}{0in}
      \setlength{\topsep}{0in} \setlength{\partopsep}{0in}
      \setlength{\leftmargin}{0.17in}}}{\end{list}}
\newenvironment{list2}{
  \begin{list}{{\small$\bullet$}}{%
      \setlength{\itemsep}{0.05in}
      \setlength{\parsep}{0in} \setlength{\parskip}{0in}
      \setlength{\topsep}{0in} \setlength{\partopsep}{0in}
      \setlength{\leftmargin}{0.2in}}}{\end{list}}


\begin{document}


\name{{\namefont CHUANMIN JIA} \vspace*{.1in}}

\begin{resume}
\section{\sc Contact Information}
\vspace{.05in}
%Centre for Multimedia and Network Technology \\
%School of Computer Engineering, Nanyang Technological University\\
%N4-B2C-06, 50 Nanyang Avenue, Singapore 639798  \\
%{\Mobilefone~+65-90884656}\\
%{\Letter~\href{mailto:wli1@e.ntu.edu.sg}{\texttt{wli1@e.ntu.edu.sg}}}
%\begin{tabular}{@{}p{4.205in}p{1.33in}}
%Centre for Multimedia and Network Technology & \rightline{\Mobilefone~+65-90884656}\\
%School of Computer Engineering & %\rightline{\Letter~\href{mailto:wli1@e.ntu.edu.sg}{\texttt{wli1@e.ntu.edu.sg}}}\\
%Nanyang Technological University \\
%N4-B2C-06, 50 Nanyang Avenue, Singapore 639798  \\
%\href{https://sites.google.com/site/xyzliwen/}{\texttt{https://sites.google.com/site/xyzliwen/}}\\
%\end{tabular}

\begin{tabular}{@{}p{4.205in}p{1.33in}}
Institute of Digital Media & \rightline{postcode 100871} \\
National Engineering Laboratory for Video Technology & \rightline{\Letter~\href{mailto:cmjia@pku.edu.cn}{\texttt{cmjia@pku.edu.cn}}}\\
School of Electronics Engineering and Computer Science \\
Peking University \\
S2728, No. 5 Yiheyuan Road, Haidian District, Beijing 100871, China \\
{\href{http://http:www.jiachuanmin.site}{\texttt{http://www.jiachuanmin.site}}}\\
{\href{https://scholar.google.com/citations?user=x5Na9n0AAAAJ&hl=zh-CN}{\texttt{Google Scholar}}}\\

%Nanyang Technological University \\
%N4-B2C-06, 50 Nanyang Avenue, Singapore 639798  \\
%\rightline{\Mobilefone~+65-90884656}\\
\end{tabular}

\section{\sc Research Interests}
\begin{list2}
\item{Image and Video Processing/Compression}
\item{Light Field Technology}
\item{Deep Learning}
\item{Computational Imaging}

%\item{Theories: Machine Learning (sparse/collaborative representation, transfer learning), Deep Learning, Computer Vision.}
%\item{Applications: Image Processing (image super-resolution, denoising), Domain Adaptation, Visual Recognition.}

%\item{Applications: text, image and video retreival/categorization}
\end{list2}

\section{\sc Education}
%2010 - Now  Nanyang Technological University (GPA 4.63)
%  I am a PhD candidate in School of Computer Engineering and my supervisor is Dong Xu.
%2007 - 2010  Beijing Normal University (BNU)
%��	  Master in Computer Science and Application (Rank 1; 92.4/100)
%2003 - 2007  Beijing Normal University (BNU)
%��	  Bachelor of Science in Computer Science and Technology (Top 5%; 86.5/100)

{\bf Peking University} \hfill {Sep. 2011 -- Current}
% \vspace{-.3cm}
\begin{list2}
%\item{{\em M.E. Candidate} in  Control Engineering  \hfill \textbf{(expected)}\qquad~~~~~}
\item{{\em Ph.D. in Computer Science} }
\item{{Advisor: Prof. Siwei Ma and Prof. Wen Gao} }
\end{list2}
\vspace{-.3cm}

{\bf Beijing University of Posts and Telecommunications } \hfill {Sep. 2011 -- Jul. 2015}
% \vspace{-.3cm}
\begin{list2}
%\item{{\em M.E. Candidate} in  Control Engineering  \hfill \textbf{(expected)}\qquad~~~~~}
\item{{\em B.Eng} in Computer Science and Technology }
\item{Overall GPA 86.7/100, rank: 10\%}
\item{Thesis: Research on Compressed Video Enhancement and GPU Acceleration. (in Chinese)}
\end{list2}
\vspace{-.3cm}


\section{\sc Research Experience}
%{\bf Nanyang Technological University, Singapore}
%
%\vspace{-.3cm}
%{\em Project Officer} \hfill Oct 2011 -- present
%
%\vspace{-.3cm} I became a research staff after my Ph.D. thesis submission. My research continues on designing machine learning algorithms (such as transfer learning, multiple instance learning, multiple kernel learning, etc.) for computer vision and text-related applications.
%

{\bf   Institute of Digital Media}, Peking University \hfill {Sep. 2014 -- present}
\begin{list2}
\item{{\em Research Assistant}}
\end{list2}
 \vspace{-.3cm}
{\bf  Institute of Computational Linguistics}, Peking University \hfill {Feb. 2014 -- Aug 2014}
\begin{list2}
\item{{\em Research Intern}}
\end{list2}
 \vspace{-.3cm}
{\bf Innovation Research Center}, Beijing Univ. of Posts. \& Telecom. \hfill {Aug. 2013 -- Mar. 2014}
\begin{list2}
\item{{\em Research Intern }}
\end{list2}
 \vspace{-.3cm}


\section{\sc Journal Publications}
% 1
\begin{enumerate}
\item{Siwei Ma, Xinfeng Zhang, Jian Zhang, \underline{\bf C. Jia}, Shiqi Wang and Wen Gao ``Nonlocal In-Loop Filter: The Way Toward Next-Generation Video Coding?," {\em {IEEE} MultiMedia} 23 (2), 16-26.}
\end{enumerate}
\vspace{.1cm}

%\begin{list2}
%\item{Yongbing Zhang, \underline{\bf Yulun Zhang$^*$}, Jian Zhang, and Qionghai Dai, ``Multiscale Adaptive Local Nonparametric Regression for Fast Single Image Super-Resolution," {\em submitted to {IEEE} Signal Process. Lett. (\textbf{SPL})}, {\bf under peer review}. ($^*$ corresponding author)}
%\end{list2}


\section{\sc Conference Publications}
\begin{enumerate}
% 1
\item{\underline{\bf C. Jia}, X. Zhang, J. Zhang, S. Wang and S. Ma, ``Deep Convolutional Network based Image Quality Enhancement for Low Bit Rate Image Compression," {\em Proc. of IEEE Visual Communications and Image Processing ({\bf VCIP})}, Chengdu, China, Nov. 2016. (Oral)}

\vspace{.1cm}
% 2
\item{J. Zhang, \underline{\bf C. Jia},  N. Zhang, S. Ma, and W. Gao, ``Structure-driven Adaptive Non-local Filter for High Efficiency Video Coding (HEVC)," {\em Proc. of IEEE Data Compression Conference ({\bf DCC})}, Snowbird, Utah, USA, Mar. 2016. (Oral) ({\bf Top Conference in Data Compression})}

\vspace{.1cm}
% 3
\item{J. Zhang, \underline{\bf C. Jia}, S. Ma, and W. Gao, ``Non-Local Structure-Based Filter for Video Coding," {\em Proc. of IEEE International Symposium on Multimedia ({\bf ISM})}, Miami, Florida, USA, Dec. 2015. (Oral)}

\end{enumerate}


\vspace{.1cm}
%\section{\sc Submissions}
%\begin{list2}
%\item{{\bf Wen Li} \emph{etal.}, ``Anonymous Submssion," {\em submitted to European Conference on Computer Vision (ECCV)}, 2014.}
%\item{Somebody, {\bf Wen Li} \emph{etal.}, ``Anonymous Submssion," {\em submitted to European Conference on Computer Vision (ECCV)}, 2014.}
%\item{Xinxing Xu$^*$, {\bf Wen Li}$^*$, Ivor W. Tsang, and Dong Xu, ``Multi-view Ambiguous Learning using Co-Labeling," {\em submitted to IEEE Transactions on Pattern Analysis and Machine Intelligence (T-PAMI)}} ($^*$ equal contributions).
%\end{list2}

%\item{Image Sampling, Coding and Reconstruction based on Compressive Sensing Theory, $\yen 190,000$, National Natural Science Foundation of China, 2016/1 �C- 2016/12.}
%\item{Conference Reviewer for ICME 2014, ICME2013, IJCAI2013, ACM-MM2013}
%\item{Journal Reviewer for International Journal of Computer Vision (IJCV), IEEE Transactions on %Neural Networks and Learning Systems (T-NNLS), IEEE Transactions on Systems, Man, and Cybernetics: %Part B (T-SMC-B), Machine Vision and Applications (MVAP), Signal Processing (SP)}


\vspace{0.8mm}
\section{\sc Honors and Awards}
\begin{list2}
%\item{National Scholarship, \hfill 2016}
\item{${1}^{st}$ prize of Video Big Data Compression Contest in the
       National Graduate Contest on Smart-City Technology. \hfill 2016}
\item{Excellent Graduation Thesis Award, Beijing Univ. of Posts. \& Telecom. \hfill 2015}
\item{Innovation Scholarship (Collaborative Innovation Center for Future Media Network). \hfill 2015}
\item{Honorable Mention Winner in American Mathematical Contest in Modeling. \hfill 2014}

\end{list2}

\section{\sc Professional Service}
\vspace{-.4cm}
\subsection{\sc Reviewer}
\begin{list2}
\item{Journal of Visual Communication and Image Representation (since Oct 2016)}

\end{list2}

\vspace{-.3cm}
\subsection{\sc Conferences (Oral or Poster) and Invited Talks}
\begin{list2}
\item{Deep Convolutional Network based Image Quality Enhancement for Low Bit Rate Image Compression, VCIP2016, Chengdu, China, Nov. 2016}

\end{list2}

\vspace{-.3cm}
\subsection{\sc Memberships}
\begin{list2}
\item{Student Member, IEEE}
\end{list2}

\section{\sc Selected Projects}
\begin{list2}
\begin{subsection}{\normalsize View Synthesis Optimization, ~~~~~~~~~~~~~~~~~~~~~~~~~~~~~~~~~~~~~~~~~~~~~~~~~~~~~~~~~~~~~~~~~~~~~~~~  Apr. 2016 - Sep. 2016}
    \item{Optimizing an open-source view synthesis software using CUDA. 6X acceleration, real time FHD (1080P) view synthesis with over 40fps.}
\end{subsection}
\begin{subsection}{\normalsize NFC Tour Guide, ~~~~~~~~~~~~~~~~~~~~~~~~~~~~~~~~~~~~~~~~~~~~~~~~~~~~~~~~~~~~~~~~~~~~~~~~~~~~~~~~~~~~~~~~  Aug. 2013 - Mar. 2014}
    \item{Developed an Android app using NFC for tourism. Mainly responsible for implementing of NFC pay, speech tour guide and database interface design.}
\end{subsection}
\begin{subsection}{\normalsize Flower Recognition, ~~~~~~~~~~~~~~~~~~~~~~~~~~~~~~~~~~~~~~~~~~~~~~~~~~~~~~~~~~~~~~~~~~~~~~~~~~~~~~~~~~~~~  Jul. 2013 - Sep. 2013}
    \item{Proposed recognition algorithm by combining histogram and contour feature with linear classifier.
          iOS app development and recognition algorithm implementation.}
\end{subsection}
\end{list2}


\section{\sc Skills}
\begin{list2}
\item{Programming: C/C++, CUDA, MATLAB, Power Shell, HTML, Python, \LaTeX.}
\item{Operating System: Mac OS X, Ubuntu Linux, Windows}
\item{Libraries/Frameworks: Caffe, OpenCV, Numpy, Tensorflow, HEVC/H.265, AVS2}
\item{Github Repo: {\href{https://github.com/codersadis}{\texttt{https://github.com/codersadis}}} }
\end{list2}



\begin{center}
\textit{Last updated: Dec.14 2016}

\end{center}
\end{resume}
\end{document}
