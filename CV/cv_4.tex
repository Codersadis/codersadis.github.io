%%%%%%%%%%%%%%%%%%%%%%%%%%%%%%%%%%%%%%%%%
% Medium Length Professional CV
% LaTeX Template
% Version 2.0 (8/5/13)
%
% This template has been downloaded from:
% http://www.LaTeXTemplates.com
%
% Original author:
% Trey Hunner (http://www.treyhunner.com/)
%
% Important note:
% This template requires the resume.cls file to be in the same directory as the
% .tex file. The resume.cls file provides the resume style used for structuring the
% document.
%
%%%%%%%%%%%%%%%%%%%%%%%%%%%%%%%%%%%%%%%%%

%----------------------------------------------------------------------------------------
%	PACKAGES AND OTHER DOCUMENT CONFIGURATIONS
%----------------------------------------------------------------------------------------

\documentclass{resume} % Use the custom resume.cls style

\usepackage[left=0.75in,top=0.6in,right=0.75in,bottom=0.6in]{geometry} % Document margins

\name{Chuanmin Jia} % Your name
\address{2728, No.2 Science Building\\ Peking University} % Your address
\address{Haidian District \\ P.R.China 100871} % Your secondary addess (optional)
\address{(010)~$\cdot$~6275~$\cdot$~6172 \\ cmjia@pku.edu.cn} % Your phone number and email

\begin{document}

%----------------------------------------------------------------------------------------
%	EDUCATION SECTION
%----------------------------------------------------------------------------------------

\begin{rSection}{Education}

{\bf Peking University} \hfill {\em Sep 2015 - Current} \\
Ph.D. in Computer Science  

{\bf Beijing University of Posts and Telecommunications} \hfill {\em September 2011 - July 2015} \\
B.E. in Computer Science \& Technology \\
GPA: {86.7/100, rank: {10\%}} \\
Thesis: (\emph {Research on Compressed Video Enhancement and GPU Acceleration})
\end{rSection}

%----------------------------------------------------------------------------------------
%	Publication SECTION
%----------------------------------------------------------------------------------------
\begin{rSection}{Publications}
\begin{rSubsection}{Journal Paper}{}{}% journal

\item Siwei Ma, Xinfeng Zhang, Jian Zhang, {\bf Chuanmin Jia}, Shiqi Wang, Wen Gao, Nonlocal In-Loop Filter: The Way Toward Next-Generation Video Coding? IEEE MultiMedia 23 (2), 16-26.

\end{rSubsection}
\begin{rSubsection}{Conference Paper}{}{}% conference

\item {\bf Chuanmin Jia}, Xiang Zhang, Shiqi Wang, Jian Zhang, Siwei Ma, Deep Convolutional Network based Image Quality Enhancement for Low Bit Rate Image Compression, IEEE Visual Communications and Image Processing (\textbf {Oral}) (Coming soon). \\

\item Jian Zhang, {\bf Chuanmin Jia}, Siwei Ma, Wen Gao, Structure-driven Adaptive Non-local Filter for High Efficiency Video Coding (HEVC), IEEE Data Compression Conference (DCC2016) (\textbf {Oral}). \\

\item Jian Zhang, {\bf Chuanmin Jia}, Siwei Ma, Wen Gao, Non-Local Structure-Based Filter for Video Coding, 2015 IEEE International Symposium on Multimedia (ISM), 301-306 (\textbf {Oral})..

\end{rSubsection}
\end{rSection}

%----------------------------------------------------------------------------------------
%	RESEARCH  INDUSTRY EXPERIENCE SECTION
%----------------------------------------------------------------------------------------

\begin{rSection}{Research}

\begin{rSubsection}{Video Coding Lab}{Sep 2015 - Present}{Research Assistant}{Peking University, Beijing}
\item Working in the area of video compression with an emphasis on in-loop filter and deep learning. Built a novel in-loop filter algorithm for High Efficiency Video Coding (HEVC) for both significant objective and subjective quality.
\item Established a novel deep convolutional network based image compression framework (on going work is optimizing). 
\end{rSubsection}

%------------------------------------------------

\begin{rSubsection}{Institute of Computational Linguistics}{Feb 2014 - Aug 2014}{Research Internship}{Peking University, Beijing}
\item Generating Chinese word embedding by Deep learning algorithm and apply word embedding into NLP tasks such as POS tagging, NER, etc.
\item Deep Learning application on Chinese word segmentation.
\item CUDA-based multi-view synthesis optimization algorithm (x6 faster than original one).
\end{rSubsection}

%------------------------------------------------

\begin{rSubsection}{Innovation research center}{Aug 2013 - Mar 2014}{Research Internship}{Beijing University of Posts and Telecommunications, Beijing}
\item Developed android app for tourism safety with NFC. It��s a component of National Undergraduate Innovation
      Project. Mainly responsible for code implementation of NFC pay, intelligent speech tour guide and database interface design.
\item As a member of flower recognition project, proposed recognition algorithm by combining histogram and contour feature with linear svm classifier. Mainly Responsible for
      iOS app development and recognition algorithm optimization..
\end{rSubsection}

\end{rSection}


%\begin{rSection}{Teaching}
%
%
%
%\begin{rSubsection}{Operating System}{January 2016 - May 2016}{Teaching Assistant}{Peking University, Beijing}
%\item Held office hours, graded projects.
%\end{rSubsection}
%
%
%\end{rSection}


%----------------------------------------------------------------------------------------
%  SKILLS SECTION
%----------------------------------------------------------------------------------------

\begin{rSection}{SKILLS}

\begin{tabular}{ @{} >{\bfseries}l @{\hspace{6ex}} l }

\textbf{ Programming Language}: C/C++, CUDA, Python, Power Shell, MATLAB, HTML \\
\textbf{ Operating System}: Mac OS X, Ubuntu Linux, Windows \\
\textbf{ Libraries/Frameworks}: Caffe, OpenCV, Numpy/Scipy, Keras, HEVC/H.265, AVS2 \\
\textbf{ Github Repository}: https://github.com/codersadis/ \\

\end{tabular}

\end{rSection}


%----------------------------------------------------------------------------------------
%	TECHNICAL STRENGTHS SECTION
%----------------------------------------------------------------------------------------

\begin{rSection}{AWARDS}

\begin{tabular}{ @{} >{\bfseries}l @{\hspace{6ex}} l }

Excellent Graduation Thesis.  ~~~~~~~~~~~~~~~~~~~~~~~~~~~~~~~~~~~~~~~~~~~~~~~~~~~~~~~~~~~~~~~~~~~~~~~~~~~~~2015 \\
Innovation Scholarship(Collaborative Innovation Center for Future Media Network).  ~~~~2015\\
Honorable Mention(The Consortium for Mathematics and Its Application (COMAP)).~~2014\\

\end{tabular}

\end{rSection}



%----------------------------------------------------------------------------------------
%  SERVICE SECTION
%----------------------------------------------------------------------------------------

\begin{rSection}{SERVICE}

\begin{tabular}{ @{} >{\bfseries}l @{\hspace{6ex}} l }

Volunteers for Beijing Garden EXPO 2013~~~~~~~~~~~~~~~~~~~~~~~~~~~~~~~~~~~~~~~~~~~~~~~~~~~~~~~~~~~~~

\end{tabular}

\end{rSection}



%----------------------------------------------------------------------------------------
%	EXAMPLE SECTION
%----------------------------------------------------------------------------------------

%\begin{rSection}{Section Name}

%Section content\ldots

%\end{rSection}

%----------------------------------------------------------------------------------------

\end{document}
